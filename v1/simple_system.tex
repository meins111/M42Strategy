%!TEX root = ./M42_Strategy.tex
\chapter{Units \& Battlegroups}
\section{Units}
Units are the building blocks of the military force of a faction. The scale of a single unit varies greatly between (and even within) factions. While the Imperial Guard fields sprawling regiments several hundreds strong, the Adeptus Astartes deploy single squad units or even a single Dreadnought. The Adeptus Mechanicus can field its Skitarii forces in a similar fashion as a Guard Regiment, but has also access to Titans, which are deployed alone or in hunting packs of two.

\subsection{Stats}
All units have two stats: HP and Morale. Both are measured in percentages.
\begin{itemize}
	\item \textbf{HP:} represents the physical status of the unit, whether it is the wellbeing of mortal troops or the hull integrity of a void ship. HP is lost during combat and recovers when the unit receives necessary supplies and the time to do so (see section \ref{recovery_section}). A units HP can never rise above 100\%. Should a units HP drop to 0\% it is destroyed.
	\item \textbf{Morale:} represents the mental status of the unit. A low morale reduces the combat effectiveness considerably, while a high morale may drive a unit to new heights. A units morale may rise above 100\% (e.g. by the skills of linked commanders and heroes). Should a units morale drop to 0\%, it will break and attempt to flee the battlefield. Depending on the outcome of the war, such a unit may be run down by the enemy, field executed by their own or actually survive to life with the shame of cowardice. 
\end{itemize}
From HP and Morale a units current Combat Effectiveness	(CE) is derived, which determines the strength of a unit during a fight.
\begin{itemize}
	\item \textbf{CE:} $HP \times \frac{Morale}{100}$
\end{itemize}


\section{Battle Groups \& Units}
A Battle Group is a collection of units, lead by a commander. Given the nature of warfare in the grim-dark future, a Battle Group is (usually) accompanied by or comprised of Imperial Navy elements, providing inter- and intra-system transport, protection and supply storage for the units making up the Battle Group.  In game, a Battle Group can be called any number of things like Army, Fleet, Crusade, or Strike Group. 
These are all well and good and their use is encouraged, but this document will stick with Battle Group for clarity.
Units are the smaller parts that make up the whole that is the Battle Group. \\
Any Battle Group is lead by a Commander and may be accompanied by a number of heroes.

\section{Commanders \& Heroes}
A commander is an accomplished leader of a faction who is put in command of a Battle Group. The faction leader of each faction is the overall commander and usually the commander of the key forces of the faction and is considered a Hero as well (see below).
Secondary Commanders can be assigned to new Battle Groups as necessary. Commanders can gain traits and followers (advisors, bodyguards, ...) during the course of their career.

Heroes are singular entities renowned throughout their factions and possibly beyond. Heroes can move around on their own - they do not need a commander or large fleet. A single small warp-capable ship will suffice.
Heroes can be linked to a Battle Group, providing a morale boost to all units therein. 
Whether or not they are linked to a Battle Group, heroes have different skills they can use to influence the world around them.
One hero can only use one active skill per round.
Similar to Commanders, heroes will acquire traits and followers throughout their career.

\subsection{Skills}
Skills appear in two variants: General (passive) and Unique (active) skills.
\begin{itemize}
	\item \textbf{General:} Skills in this category are passive in nature and (usually) revolve around strengthening own units, speeding up (re-)deployment or reducing travel time. Secondary Commanders and heroes start with one randomly determined General skill. Faction leaders start with two. The following list offers a selection of the most common General skills available, but the GM and players are encouraged to come up with additional ones.
	\begin{itemize}
		\item \textbf{Command Authority:} Units of the associated Battle Group have a slightly increased morale (10\%) and take no CE penalty when cooperating with allied forces in the same battlefield.
		\item \textbf{Cooperative Forces:} The Battle Group has an increased CE when fielded alongside allied units.
		\item \textbf{Rapid Relocation:} Efficient protocols reduce the (re-)deployment delay of the Battle Group by one round.
		\item \textbf{Omnipresent Logistics:} Battle Group units have a 10\% increased recovery rate.
		\item \textbf{Crash Course Recruitment:} Unit recruitment in the local system is reduced by 25\% to a minimum of one round.
	\end{itemize}
	\item \textbf{Unique:} Skill of this group are unique to a character, allowing him to influence his environment in a specific and active manner. Secondary Commanders and Heroes start with one such skill, while faction leaders start with two. The following are examples to spark the creativity of GM and players alike.
	\begin{itemize}
		\item \textbf{Pulled from the Brink:} The character is determined to not let any void ship go to waste even if others consider it beyond salvation. He can spend a full round to analyze the hulks of a void battle in the order to seek out salvageable wreckage. Such ships are considered at 0\% HP and can be repaired according to the normal rules (see section \ref{recovery_section}). He has also a chance to find usable ship parts when doing so.
		\item \textbf{Screening Protocol:} The Apothecari routinely screens the population of visited worlds to find suitable candidates for Astartes initiation. When performing this action for one or more rounds, he has a chance to detect genestealer infestations, mutations and diseases in the population in addition to a chance of finding new Neophytes for his chapter.
		\item \textbf{Artificer:} Given enough time and access to resources, the Master of the Forge (Tech Marine Hero) is able to craft even the most advanced war-gear of the chapter, including Terminator armor (\textit{Part, Chapter Hold, 8 rounds}) and Dreadnought chassis (\textit{Chapter Hold, 16 rounds}). This action requires the Battle Group or chapter hold to use the Produce stance (see section \ref{stances}) but does not count against its production slots.
		\item \textbf{Purge:} The Commissar is signing a Battle Group or even system wide order to purge the unclean. Any unit of the Battle Group that has one of the following traits will suffer HP damage and looses the trait: Infected, Infested (Genestealers), Mutated, Poisoned. A system wide order will inflict HP damage to a planet and has a chance to fully remove the trait from the local population.
	\end{itemize}
\end{itemize}

\subsection{Traits \& Followers}
TODO


\chapter{Supplies \& Production}
\section{Supplies}
Fleets, planets and Battle Groups require supplies to keep fighting and producing. 
A simplified system of Supply Points (SP) representing all resources will be used to reduce complexity at the outset.
Depending on the stance the unit or planet assumes, it may increase or lower its current supply points by a certain amount.
Should an entity run out of supply points, it will start taking CE damage each round in addition to a serious decline in effective fighting power should it be attacked by the enemy.


\section{Production}
Planets throughout the Imperium are geared towards the production of a wild variety of goods required by its fighting forces be it ammunition, tanks, fuel or new recruits.
Planets (and other production facilities) can assume the production stance to unlock the full potential of its manufactoria.
Taking on that stance opens the production slots of the entity, which can be either used to restock supply points or to recruit new units.
Production slots are divided into three levels, representing the technology required to produce goods of that level.
\begin{itemize}
	\item \textbf{Minima:} Basic goods that can be produced on nearly any Imperial planet. One Minima slot can produce one SP per round.
	\item \textbf{Vexillus:} Industrial-grade goods which can be produced on most civilized worlds of the Imperium but which are not available on more archaic worlds.	One Vexillus slot can produce two SP round.
	\item \textbf{Exactus:} Highly advanced technology that can only be produced on forge worlds or a few other places with a high density of high-ranking Tech priests. An Exactus slot can produce three SP per round.
\end{itemize}

\subsection{Logistic Groups}
Very large fleets are usually accompanied by at least one but possibly several ship groups tasked with producing the most important goods (food, ammunition) on-the-fly, expecially if supply lines are expected to be threatened or unreliable. 
Such groups usually have a Goliath-class factory ship at their heart as well as a host of smaller vessels tasked with mining asteroids, gas nebulae, and stellar winds for base resources which are then processed by the Goliath into ammunition, wargear, tools, armor plating and most other industrial-grade goods the fleet requires. \newline
A single logistic group can store six SP and has 2 Minima and 2 Vexillus production slots if it assumes the Production stance. Thus, a single logistic group can produce a up to a total of 6 SP per round.
\begin{itemize}
	\item \textbf{Astartes Chapter Fleet:} Start with two logistic groups.
	\item \textbf{Ad-Mech Explorator Fleet:} Start with two logistic groups and may include an Ark Mechanicus, which has the following profile. Slots: 2$\|$2$\|$2 (up to 12 SP).
\end{itemize}

\chapter{Combat}
\section{Stances} \label{stances}
The exact upkeep cost of an entity is decided by the Stance(s) it is assuming during a round. Five types of Stances are defined here. Some of these can be combined. In such a case, add together the upkeep cost of each activity to get the total upkeep. If two Stances reduce Warp travel, that penalty does not stack.

\begin{itemize}
	\item \textbf{Idle:} This is the most basic "activity" possible and is generally frowned upon. This is only acceptable during times of severe resource shortage. Entities in this stance try to keep their upkeep as low as possible. Units are confined to their base without field training, planets stop their industry and send their workers home and fleets are put into high anchor or a stable orbit.
	\begin{itemize}
		\item \textbf{Cost:} 0 SP.
		\item Planets and other producers loose their basic production and halt any projects and recruitments.
		\item Units loose 1 experience point per round while staying idle.
		\item This activity cannot be combined with other activities.
	\end{itemize}

	\item \textbf{Move:} Moving a single star ship through the void is a costly endeavor, but moving an entire fleet with accompanying ground forces is a herculean effort.
	\begin{itemize}
		\item \textbf{Cost:} 1 SP.
		\item Unlocks inter-system warp travel and large scale intra-system ground force redeployment.
		\item This can be combined with: Produce, Brace and Recover. If combined with Produce, reduce the number of hexes you can travel by half rounded down.
	\end{itemize}

	
	\item \textbf{Produce:} This activity represents the expenditure of resources to produce something (usually supplies, a unit, or a planetary advancement). See table \ref{production_table_units} for example unit production projects.
	\begin{itemize}
		\item \textbf{Cost:} 2 SP.
		\item Unlocks the production slots of the entity (if any).
		\item Unlocks the basic production of the entity (if any).
	\end{itemize}
	
	\item \textbf{Brace:} Being prepared for war can be a matter of life and death in the endless wars of the 42nd millennium. Units have to perform regular live-fire drills and regular combat exercises to keep their edge. Planetary defense forces need regular drills as well. Walls and gun emplacements need to be repaired and calibrated. Fleets have to patrol and perform maneuvers. Orbital defenses require regular maintenance and calibration just like ground emplacements.
	\begin{itemize}
		\item \textbf{Cost:} 2 SP.
		\item Units gains one experience point per turn
		\item Non-Unit entities (e.g. planets) can raise their CE beyond 100\% by taking this action. It rises by 10\% per round to a maximum of 150\%.
		\item Units taking this activity gain the 'on guard' trait for the duration of the activity thus raising the chance to detect enemy activity in the local area.
		\item Can be combined with the Move, Recover and Produce activities. 
		\item If combining this stance with Produce, non-unit entities reduce their base production and available production slots by half (round up).
	\end{itemize}
	
	\item \textbf{Fight:} In the far future, there is only war. While not entirely accurate for every world in the Imperium, it certainly is for the Imperium at large. War is an unmatched devourer of resources such as food, medical supplies, ammunition, soldiers, war-gear; all are spent in vast quantities every day to keep the Imperium intact for just one more day. 
	This activity represents a major military conflict, leaving no time for anything else besides. Minor military activity such as guard duty, light skirmishes, or patrolling fall under Brace.
	\begin{itemize}
		\item \textbf{Cost:} 3 SP.
		\item Units gain experience from combat. This can range from 2 to 8 experience points depending on the scale and length of the fighting as well as the outcome - victory yields more than defeat.
		\item Units cannot combine this stance with any other.
		\item Non-Unit entities can combine this stance with any other as long as its CE remains above 50\% with the following constraints: 
		\begin{itemize}
			\item Does no longer benefit from trade routes.
			\item Production slots and base production are halved (round down)
			\item Any recruitment process has a 33\% chance to not make any process during any given round due to enemy intervention.
			\item CE increase from the Brace and Recover stance are reduced by half (round down).
		\end{itemize}
	\end{itemize}
	
	\item \textbf{Recover:} Combat losses and collateral damage are inevitable in the struggle for mankind's survival. Restoring good order, repairing damaged war-gear, and filling up the losses with new recruits is the purview of the Recover activity. The resource demands between two given units can be very large, thus the abstract notion of recovery points (RP) is used. The higher the RP of a unit type, the higher the cost to recover it.
	\begin{itemize}
		\item \textbf{Cost:} 1 SP per five recovery points used (round up).
		\item Allows the entity in question to restore CE. Note that advanced units may have additional requirements that must be fulfilled before recovery kicks in (e.g. some may require a certain tech-level to be met).
		\item Recovering causes the unit to loose some experience points as new recruits fill up the ranks of dead veterans. For mechanical units, this represents the introduction of new parts (and thus machine spirits) into a greater whole, introducing minute changes in its behavior. A unit looses one experience for each 20\% CE it recovers.
		\item Non-Unit entities recover 10\% per turn.
		\item Can be combined with the Move, Brace and Produce activities.
	\end{itemize}   
\end{itemize}

\section{Recovery} \label{recovery_section}
Table \ref{recovery_table} provides an overview over the requirements and resource demands of a unit recovering from combat damage.

\onecolumn
\chapter{Tables}
\begin{center}
 \begin{longtable}{l l l l l}\toprule
	Unit Name & CE & Requirement & RP & Rate \\ \endhead\midrule
	Regular Infantry & any & Minima & 0.5 & 40\% \\
	Special Infantry & any & Minima & 1 & 33\% \\
	Mechanized & any & Vexillus & 1 & 33\% \\
	Armored & any & Vexillus & 2 & 25\% \\
	\makecell[cl]{Astartes} & $\ge$75\% & - & 0.5 & 5\%\\
	\makecell[cl]{Astartes} & $<$75\% & - & - & 0\%\\  
	\makecell[cl]{Astartes} & $\ge$50\% & Hold or Fleet & 1 & 40\%\\
	\makecell[cl]{Astartes} & $<$50\% & Hold or Fleet & - & special\footnote{Needs recruits from the next lower experience level.}\\
	Knight & $\ge$50\% & \makecell[lc]{Vexillus or\\Knight World} & 1 & 25\%\\
	Knight & $<$50\% & \makecell[lc]{Exactus or\\Knight World} & 2 & 20\%\\
	Titan & $\ge$75\% & Vexillus & 3 & 15\% \\
	Titan & $\ge$50\% & Exactus & 4 & 15\% \\
	Titan & $<$50\% & Exactus & 4\footnote{Plus one or more randomly-determined parts depending on the severity of damage suffered.} & 15\% \\
	Void Ship & $\ge$75\% & - & 2 & 5\%\\
	Void Ship & $\ge$75\% & Space Port & 2 & 25\%\\
	Void Ship & $\ge$50\% & \makecell[cl]{Vexillus,\\Space Port} & 4 & 20\%\\
	Void Ship & $\ge$25\% & \makecell[cl]{Vexillus,\\Space Port} & 4\footnote{Plus one randomly-determined part} & 10\%\\
	Void Ship & $<$25\% & \makecell[cl]{Exactus} & 5\footnote{Plus two or more randomly-determined parts}  & 10\%\\
	\bottomrule
	\caption{Recovery table by unit type.\label{recovery_table}}\\
\end{longtable}	
\end{center}

\begin{center}
\begin{longtable}{l l c c c } \toprule
    Name & Type & Tech Level & Time & Parts \\ \midrule\endhead
    \makecell[cl]{Basic Infantry} & Basic & Minima & 1 & - \\ \addlinespace
    \makecell[cl]{Mechanized Infantry} & Advanced & Vexillus & 3 & - \\ \addlinespace
    \makecell[cl]{Special Infantry} & Advanced & Vexillus & 3 & - \\ \addlinespace
    \makecell[cl]{Pulled Artillery} & Advanced & Vexillus & 3 & - \\ \addlinespace
    \makecell[cl]{Leman Russ Hull}  & Part & Vexillus & 2 & - \\ \addlinespace
    \makecell[cl]{Turret (Basic)} & Part & Vexillus & 2 & - \\ \addlinespace
    \makecell[cl]{Turret (Advanced)}  & Part & Exactus & 3 & - \\ \addlinespace
    \makecell[cl]{Leman Russ Assembly}  & Assembly & Vexillus & 1 & \makecell[cl]{Hull,\\Turret} \\\addlinespace
    \makecell[cl]{Astartes Scouts} & Advanced & Minima & 2 & Neophytes \\ \addlinespace
    \makecell[cl]{Astartes Wargear} & Part & Exactus & 4 & - \\ \addlinespace
    \makecell[cl]{Knight Chassis} & Part & Vexillus & 1 & - \\ \addlinespace
    \makecell[cl]{Knight Power Core} & Part & Exactus & 2 & - \\\addlinespace
    \makecell[cl]{Titan Basic Weapon} & Part & Vexillus & 1 & - \\ \addlinespace
    \makecell[cl]{Titan Advanced Weapon} & Part & Exactus & 2 & - \\ \addlinespace
    \makecell[cl]{Knight} & Assembly & Vexillus & \makecell{1 per\\part} & \makecell[cl]{Chassis,\\Core,\\Weapons} \\ \addlinespace
    \makecell[cl]{Warhound Chassis} & Part & Vexillus & 4 & - \\ \addlinespace
    \makecell[cl]{Warhound Power Core} & Part & Exactus & 4 & - \\\addlinespace
    \makecell[cl]{Warhound} & Assembly & Exactus & \makecell{1 per\\part} & \makecell[cl]{Chassis,\\Core,\\Weapons} \\ \addlinespace
    \makecell[cl]{Warlord Chassis} & Part & Vexillus & 4 & - \\ \addlinespace
    \makecell[cl]{Warlord Legs} & Part & Vexillus & 2 & - \\ \addlinespace
    \makecell[cl]{Warlord Head} & Part & Exactus & 4 & - \\ \addlinespace
    \makecell[cl]{Warlord Power Core} & Part & Exactus & 6 & - \\\addlinespace
    \makecell[cl]{Warlord} & Assembly & Exactus & \makecell{1 per\\part} & \makecell[cl]{Chassis,\\Core,\\Legs,\\Head,\\Weapons}\\ \addlinespace
    \makecell[cl]{Ship Hull Section} & Part & Vexillus & 4 & - \\ \addlinespace    
    \makecell[cl]{Ship Core} & Part & Exactus & 6 & - \\ \addlinespace  
    \makecell[cl]{Ship Bridge} & Part & Exactus & 4 & - \\ \addlinespace 
    \makecell[cl]{Ship Engines} & Part & Exactus & 4 & - \\ \addlinespace 
    \makecell[cl]{Macrocannon Battery} & Part & Vexillus & 2 & - \\ \addlinespace 
    \makecell[cl]{Lance Battery} & Part & Exactus & 3 & - \\ \addlinespace 
    \makecell[cl]{Torpedo Tubes} & Part & Vexillus & \makecell{1 per\\2 tubes} & - \\ \addlinespace 
    \makecell[cl]{Flight Deck} & Part & Vexillus & 6 & - \\ \addlinespace 
    Frigate & Assembly & Vexillus & \makecell{1 per\\part}& \makecell{Hull,\\Core\\Engine\\Weapons} \\\addlinespace
    Cruiser & Assembly & Exactus & \makecell{1 per\\part}& \makecell{2xHull,\\2xCore\\Engine\\Bridge\\Weapons} \\\addlinespace
    Battleship & Assembly & Exactus & \makecell{1 per\\part}& \makecell{4xHull,\\2xCore\\2xEngine\\Bridge\\Weapons} \\\addlinespace
     
        \bottomrule
        \caption{Unit production table.\label{production_table_units}}\\
\end{longtable}
\end{center}

\clearpage
\twocolumn