%!TEX root = ./M42_Strategy.tex
\chapter{Supplies \& Production}
\textit{You can never have enough ammunition, fuel, canned rations and toilet paper. \\- Unknown Commander, ca. M02}
Units, fleets and planets (summarily called Entity in this chapter) require supplies to keep fighting and producing. 
A simplified system of Supply Points (SP) representing all resources will be used to reduce complexity at the outset.
Depending on the stance the unit or planet assumes, it may increase or lower its current supply points by a certain amount.
Should an entity run out of supply points, it will start taking HP and Morale damage each round in addition to an additional CE penalty should it be attacked by the enemy.

\section{Production}
Planets throughout the Imperium are geared towards the production of a wild variety of goods required by its fighting forces be it ammunition, tanks, fuel or new recruits.
Planets (and other production facilities) can assume the production stance to unlock the full potential of its manufactoria.
Taking on that stance opens the production slots of the entity, which can be either used to restock supply points or to recruit new units.

\section{Technology Levels}
Production slots are divided into three levels, representing the level of infrastructure required to produce goods of that level.
\begin{itemize}
	\item \textbf{Minima:} Basic goods that can be produced on nearly any Imperial planet, relying mostly on manual labor and ubiquitously available tools. One Minima slot can produce one SP per round.
	\item \textbf{Vexillus:} Large-scale manufactoria powered by engines of steam, fuel, electricity, plasma generators or solar collectors dot the surface of many civilized and Hive worlds of the Imperium. The majority of common products is mass produced in such places. One Vexillus slot can produce two SP round.
	\item \textbf{Exactus:} The most advanced technology and the ways to produce them is a highly guarded secret of the highest echelons of the Adeptus Mechanicus and limited to their own manufactoria, usually reserved for Forge Worlds. An Exactus slot can produce three SP per round.
\end{itemize}

\section{Tithes}
Nearly every single planet in the Imperium of Men is subject to tithes collected by the Adeptus Terra. 
The type and quantity of tithes vary as wildly as the range of worlds that make up the Imperum vary themselves.
Commonly tithes come in the form of locally produced goods or military personnel for the endless wars of the Imperium.\\
In gameplay terms, every planet must use some of its yearly production capabilities to satisfy its tithes. The exact tithes are noted for each planet. 
Planets associated with the players have reduced tithe levels, representing the Imperium recognizing and supporting the war efforts of the respective faction by allowing them greater influence over the industrial output of their planets.
\begin{DndSidebar}{Insol-8 [Planet]}
\begin{itemize}
	\item \textbf{Sub-Sector:} Hastea
	\item \textbf{Classification:} Mining / Hive World
	\item \textbf{Tech-Level:} Vexillus
	\item \textbf{Slots:} 2 | 1 | 0
	\item \textbf{Storage:} 12 SP
	\item \textbf{Tithes:} Annually:
	\begin{itemize}
		\item Ore \& Mineral Shipment: equivalent to 4 SP
		\item One Guard Regiment (Regular Infantry)
	\end{itemize}
\end{itemize}
\end{DndSidebar}

\chapter{Units \& Battlegroups}
\section{Units}
Units are the building blocks of the military force of a faction. The scale of a single unit varies greatly between (and even within) factions. While the Imperial Guard fields sprawling regiments several hundreds strong, the Adeptus Astartes deploy single squad units or even a single Dreadnought. The Adeptus Mechanicus can field its Skitarii forces in a similar fashion as a Guard Regiment, but has also access to Titans, which are deployed alone or in hunting packs of two.

\subsection{Stats}
All units have two stats: HP and Morale. Both are measured in percentages.
\begin{itemize}
	\item \textbf{HP:} represents the physical status of the unit, whether it is the wellbeing of mortal troops or the hull integrity of a void ship. HP is lost during combat and recovers when the unit receives necessary supplies and the time to do so (see section \ref{recovery_section}). A units HP can never rise above 100\%. Should a units HP drop to 0\% it is destroyed.
	\item \textbf{Morale:} represents the mental status of the unit. A low morale reduces the combat effectiveness considerably, while a high morale may drive a unit to new heights. A units morale may rise above 100\% (e.g. by the skills of linked commanders and heroes). Should a units morale drop to 0\%, it will break and attempt to flee the battlefield. Depending on the outcome of the war, such a unit may be run down by the enemy, field executed by their own or actually survive to life with the shame of cowardice. 
\end{itemize}
From HP and Morale a units current Combat Effectiveness	(CE) is derived, which determines the strength of a unit during a fight.
\begin{itemize}
	\item \textbf{CE:} $HP \times \frac{Morale}{100}$
\end{itemize}

\subsection{Power Tiers}
Units are divided up into several tiers representing their baseline strength. 
Units within the same tier are roughly on equal terms. 
Units of the same tier prefer to fight enemy units of the same tier.
Should a unit fight an enemy of a lower tier it gains two Edges (plus one for each tier difference beyond the first) and increase any inflicted damage by half.
\begin{itemize}
	\item \textbf{Tier 1 - Swarms:} Masses of low quality troops are not a danger to most other units by themselves but usually come with overwhelming numbers. Example units: Militia, Flagellates, Gaunts, chaos spawn, Nurglings, Gretchins, Squigs, cultists.
	\item \textbf{Tier 2 - Regular Infantry:} Units of this category represent the backbone of a factions fighting forces. They are equipped with mass-produced arms and armor as well as decent military training according to the standards of their culture. Examples: baseline Guard Regiment, PDF, baseline Skitarii, Ork Boy Mob, Fire Warriors w/o Drones, Traitor Guard Regiment, Genestealer Hybrids
	\item \textbf{Tier 3 - Special Infantry:} These units receive advanced training and equipment, elevating them above their comrades in fighting power. Examples: Storm Troopers, elite Guard Regiment (Drop Troops, Kasarkin, Kriegers, ...), advanced Skitarii, Heavy Combat Servitors, Adeptus Sororitas, Ork Kammandos, Stormboyz, Tankbustas, Fire Warrior w. Drones, Eldar Aspect Warriors, Lesser Daemons
	\item \textbf{Tier 4a - Elite Infantry:} These units represent the pinnacle of infantry-grade units of a faction. Examples: Space Marines, Assassins, Chaos Marines, Ork Nobz, Tyranid Warrior Forms, Pure-strain Genestealer, Eldar Harlequins, Wrath Guard, Necron Warriors
	\item \textbf{Tier 4b - Mechanized Infantry:} Infantry units deploying supportive armored vehicles for save and fast transport as well as for heavy fire support. Example units: Mechanized Guard Regiment (Chimera, Hellhounds, Sentinels), Astartes Bikes and Land Speeder, Eldar Jetbikes.
	\item \textbf{Tier 5a - Armor:} Mainline battle tanks in orderly ranks. Examples: Leman Russ Armored Regiments, Astartes Armor Support, Loota-Tank, Defiler, Falcon
	\item \textbf{Tier 5b - Combat Walkers:} These heavily armed and armored walkers are able to lay down and withstand withering fire and are often exceedingly deadly in close combat, especially against enemy armor. Examples: Dreadnoughts, Deaf Dread, Wraithlord, Carnifex, Terminators, Tau Battle Suites, Decimator
	\item \textbf{Tier 6a - Heavy Armor:} Bigger, slower and even stronger armored and armed than common tanks, these units are land behemoths, crushing infantry beneath their treats and laying waste to entire frontlines. Examples: Baneblade, Land Raider, Battlewagon, Brass Scorpion, Forgefiend, Hive Tyrant, Cobra,
	\item \textbf{Tier 6b - Knights:} These units are in essence upscaled combat walkers the size of small buildings and with weaponry to destroy entire colomns of tanks. Examples: Imperial Knights, Chaos Knights, Wraithknight, Greater Daemon
	\item \textbf{Tier 7 - Scout Titans:} The first true category of godmachines, scout class titans range ahead of their larger kin, lay in ambush and take on the flanks and backs of enemy titan formations. Against any non-titan unit, its weaponry is downright devastating. Examples: Warhound, Death Wheel, Stompa, Feral, Revenant, Vituperator
	\item \textbf{Tier 8 - Battle Titans:} True titans of the battlefield, these war machines are city killers and planet razors. Examples: Warlord, Reiver, Phantom, Gargant, Ravager, Bane/Plague/Pain/Warplord, Hierophant
	\item \textbf{Tier 9 - Apex:} The most devastating of them all. Examples: Imperator Titan, Ordinatus Warmachines, Warmonger, Mega-Gargant, Warlock, Hydraphant, Daemon Prince
\end{itemize}

\subsection{Experience}
Units accumulate experience when they train or fight and will loose XP when staying idle or recover from combat losses. 
At certain XP thresholds (levels) a unit will receive a new trait of equal name, representing its veteran status. 
Reaching a new level resets the accumulated XP of the unit.
Once reaching a new level, it will not drop below this level due to XP loss.
The four ranks are:
\begin{itemize}
	\item \textbf{Conscript:} Freshly and hastily recruited troops start at this rank. Militias of planets fall under this category as well. Conscript troops have a maximum morale stat of 90\%.
	\item \textbf{Soldier:} Thoroughly trained troops will start at this level. A conscript troop will require 4 XP to achieve this level. Soldier rank units can have a maximum morale stat of 120\%.
	\item \textbf{Veteran:} After long campaigning a unit may achieve the rank of veteran. Veteran units can have a maximum morale of 150\%. Soldier rank units will require 16 XP to achieve Veteran status.
	\item \textbf{Hero:} Only few units are hardy enough to become true heroes of the Imperium. Those that do are rightfully famous and feared by the enemies of mankind. Heroes can have a maximum morale stat of 200\% and will not break once reaching 0\% morale. Once a unit reaches Hero status, the controlling faction has the unique opportunity to recruit a new hero or commander from their ranks. Veterans need to accumulate 32 XP to become Heroes.
\end{itemize}

\subsection{Unit Traits}
Units will have a short list of traits, which depend on their origin, type, experience level, equipment and external circumstances.
Traits are used to determine a units fighting power in a specific circumstance. 
Most provide the unit with either an Edge or a Flaw. 
These are used to compare the combat performance of similar units.
Whichever unit has more edges than his opponent will win a confrontation.
Having a flaw adds one Edge to the total sum of the enemy unit.
Some traits have two levels: expert and master. 
Expert traits provide one Edge while master level traits provide two.
\begin{itemize}
	\item \textbf{Environmental Sealed:} Unit is immune to environmental effects, like poisonous attacks or toxic atmospheres and can be deployed in places without breathable atmosphere.
	\item \textbf{Shock Troop (Expert/Master):} The unit employs powerful shock and awe tactics or is otherwise able to inspire fear in enemy units. It gains Edge against regular infantry units and any unit with the [Expandable] trait. In addition, any inflicted Morale damage is doubled.
	\item \textbf{Anti-Infantry:} The unit is particularly equipped and trained to take out large amounts of enemy infantry but is lacking against heavily armored targets. It gains Edge against infantry units and two Edges against units with the [Strength in Number] trait but takes a Flaw against armored units. It will target swarms first followed by regular and special infantry.
	\item \textbf{Anti-Armor:} The unit is particularly equipped and trained to take out enemy armor but is lacking against large formations of lesser targets. The unit has Edge against armored targets but takes Flaw against units with the [Strength in Number] trait. It will target armored units first.
	\item \textbf{Stealthy:} The unit is adept at moving, deploying and fighting without being noticed. During ongoing wars or skirmishing this unit has a chance to deal major HP and Morale damage to a single enemy unit.
	\item \textbf{Know No Fear:} This unit will never break due to Morale loss and will continue fighting unto death. Once its morale drops to 0\% it will take 50\% increased HP damage.
	\item \textbf{Hit'n'Run:} This fast unit is able to hit an enemy hard and fast and quickly disengage before reinforcement or heavy weapons can be broad to bear on them. This unit has Edge during the first round of an engagement and during skirmishes.
 	\item \textbf{Psyker (Delta/Beta/Alpha/Alpha+):} This units has psychic powers it can field for devastating effects. It has Edge against non-psyker units but at the (low) risk of suffering serious HP damage and the even lower chance of triggering dangerous warp-related events. Alpha and Alpha+ grade Psyker deal Morale damage to the entire enemy Battle Group and gain double Edge against non-psykers but also are at a risk of triggering more dangerous events.
 	\item \textbf{Close Quarter Experts/Master:} This unit is deadly in close quarter fighting common to battles taking place in enclosed spaces (cities, caves, space ships/stations, etc). Gains Edge against units lacking this trait. Can be activated in non-enclosed combat situations if combined with certain other traits like [Drop Troop] or [Stealthy] or if supported by other units.
 	\item \textbf{Drop Troop:} The unit is able to perform combat drops from orbit or high altitude fliers right into the midst of the enemy. Doubles Morale and HP damage during the first round of combat. Allows the unit to ignore the [Lengthy Deployment] trait for the insertion (but not for redeployment).
 	\item \textbf{Environment X Training:} This unit is a master in fighting in specific combat circumstances, usually due to prolonged training or ancestral habits. Unit gains Edge and an increased base Morale by 25\% if fighting in the environment but takes Flaw in radically different terrains.
 	\begin{itemize}
 		\item Desert Edge: Snow + Urban Flaw
 		\item Snow Edge: Desert + Urban Flaw
 		\item Jungle Edge: Wasteland + Urban Flaw
 		\item Wasteland Edge: Jungle + Urban Flaw
 		\item Plain Edge: Mountainside + Urban Flaw
 		\item Mountainside Edge: Plain + Urban Flaw
 		\item Urban/Enclosed Edge: Nature Flaw
 	\end{itemize}
 	\item \textbf{Adaptable:} This unit is quick to adapt to new combat circumstances. It is treated as having the fitting Environment Training trait after the first round of fighting in a new environment. It replaces the so gained trait once it changes to another environment.
 	\item \textbf{Siege Experts/Master:} This unit is used to endure the long, arduous grind that is siege and trench warfare. Once dug in (takes one round of combat) it takes reduced HP and Morale damage from ongoing wars and units with the [Long Range] trait.
 	\item \textbf{Expandable:} This unit is of low quality or otherwise considered expandable by its own command. Its destruction or flight does not inflict morale damage to the Battle Group it is part of. Lowers maximum morale by 20\%.
 	\item \textbf{Lengthy Deployment (X):} This unit requires a complex logistic process to deploy to a new battlefield. It can only take combat actions after X rounds have passed and requires a secured beachhead. It requires half that time (round up) for redeployment from a battlefield. 
 	\item \textbf{Special Ammunition:} This unit requires especially rare and expensive ammunition (or a particularly massive amounts of a common one). It increases the upkeep cost of the Battle Group during warfare by one. The increase from this trait does not stack.
 	\item \textbf{Witch Hunter:} The unit specializes in hunting down enemy psykers. Double Edge against units with the [Psyker] trait. It will target enemy units with the [Psyker] or [Warp Entity] traits first.
 	\item \textbf{Shield Breaker:} This unit deploys weapons capable of bringing down even large void shields commonly employed by void ships, titans and fortresses.
 	\item \textbf{Carrier Craft:} This void ship is capable of fielding entire squadrons of bombers, fighters and boarding craft. It can assist in ground warfare unlike most other void ships but may take HP damage doing so - depicting the loss of fighter craft.
 	\item \textbf{Long Range:} This unit is equipped with particular long ranged weaponry (sniper rifles, artillery, missiles, lances, etc). During open field combat it has the chance to inflict HP and Morale damage to the enemy without suffering return damage. If caught in a close quarter conflict, this unit has Flaw.
 	\item \textbf{Massive Frame:} This unit is so large that boarding it and conquering (or at least damaging) it from within is a possibility. Boarding units can be infantry only and will take HP damage from doing so due to internal defense systems but are protected from this units main weaponry. The problem of course is getting infantry into the massive frame in the first place...
 	\item \textbf{Strength in Number:} This unit consist of huge numbers of individually weak units that seek to overwhelm the foe with their sheer mass. Loosing a few of its number does not faze it or its deadliness. It is immune to morale if its HP is above 50\% but takes double Morale damage if below 50\%. If outnumbering an enemy unit (that is multiple units with this trait vs one enemy unit) the units base power level increases by 1 per supporting unit (max: Base+3) but all take double HP damage. It also gains Edge against Elite Infantry units.
 	\item \textbf{Shadow in the Warp:} This unit projects the fearsome Shadow in the Warp which is notorious for its ability to shut down or at least hamper psychic powers of the enemy. Any enemy unit with the [Psyker Delta] or [Psyker Beta] trait will temporarily loose it. [Psyker Alpha] and [Psyker Alpha+] units will have their trait ranking reduced by two steps instead, so Alpha is reduced to Delta and Alpha+ becomes Beta. Warp flight from and to the system of this unit is slowed and incurs a greatly increased dangers of becoming lost in the warp. Local Astropathic communication is also distorted or completely disabled.
 	\item \textbf{Waaagh! (XY\%):} Large Ork Battle Groups have a unique and shared Morale among all its units and are thus highly resistant to Morale damage. Loss of the current Waaagh-Boss will disable this trait for a random number of rounds until a new Boss rises. Ork units with this trait disabled take double Morale damage.
 	\item \textbf{Warp Entity:} This unit is at least partially neverborn. It is immediately destroyed if its morale drops to 0\%. It gains Edge against any unit without the [Psyker] trait and inflicts 50\% increased Morale damage. It will target enemy units with the [Psyker] trait above any other. Regardless of the battle outcome, it has a low chance to add the [Tainted] hidden trait to an enemy unit.
 	\item \textbf{Favored Enemy (X):} This unit has intense training and experience in fighting a certain enemy of mankind. It gains Edge and a morale increases by 30\% when facing units of that faction. It will target units of that faction first in case of a multi-faction war-zone.
 	\item \textbf{Shielded (Warp/Energy/Void):} This unit benefits from some type of shielding. It gains double Edge against any unit without the [Shield Breaker] trait and it has a good chance to suffer no HP damage from such units. Warp shields offer no protection against attacks made from units with the [Psyker] trait.
\end{itemize}

 	\subsection{Hidden Traits}
 	Several unit traits are hidden until uncovered by certain events as well as through traits and actions of Commanders and Heroes.
 	\begin{itemize}
 	\item \textbf{Tainted:} The unit is tainted by the touch of the warp. Its maximum Morale is reduced by 30\%. Whenever it drops below 25\% Morale this trait will be replaced with [Corrupted].
 	\item \textbf{Corrupted:} A corrupted unit starts to act according to the plans of the chaos gods. It will regularly inflict HP and Morale damage to all units of the Battle Group and has a low chance to apply the [Tainted] trait a random unit of the battlegroup.
 	\item \textbf{Possessed (Nurgle/ Tzeench/ Korne/ Slanesh/ Undivided):} The unit is possessed by a daemon. It is a hidden enemy through and through. Should the accompanying Battle Group do a warp travel, chances for a dangerous warp travel event increase by 50\%. Should the Battle Group fight against a major Chaos force aligned to the same god of the possessing daemon, it will turn sides openly and inflict major Morale damage to the entire Battle Group in doing so.
 	\item \textbf{Infected:} This unit suffers from a hidden sickness. It may be one of Nurgles \textit{gifts} or the even worse fate of being infected by a genestealer. Reduces maximum morale by 25\% per year. Once its maximum morale reaches 0\% it will succumb to the infection thus either turning traitor or dying (and maybe rising as a zombie). Every round, this unit has a low chance to spreading the infection to another unit of the Battle Group or the local population.
\end{itemize}

\section{Battle Groups \& Units}
A Battle Group is a collection of units, lead by a commander. Given the nature of warfare in the grim-dark future, a Battle Group is (usually) accompanied by or comprised of Imperial Navy elements, providing inter- and intra-system transport, protection and supply storage for the units making up the Battle Group.  In game, a Battle Group can be called any number of things like Army, Fleet, Crusade, or Strike Group. 
These are all well and good and their use is encouraged, but this document will stick with Battle Group for clarity.
Units are the smaller parts that make up the whole that is the Battle Group. \\
Any Battle Group is lead by a Commander and may be accompanied by a number of heroes.

\subsection{Logistic Groups}
Crusading fleets are usually accompanied by at least one but possibly several ship groups tasked with collecting or producing raw resources (e.g. asteroid mining vessels, debris fields collector, gas planet extractors, hydroponic bays) and processing them to the most commonly required supply goods required by the fleet: food, fuel and ammunition.
In addition to these host of smaller ships, large fleets may also possess a Goliath-class factory ship (or another massive ship with considerable industrial plants installed on it) which can even produce replacement parts or even some standard patterns of vehicles and equipment or any other commonly used industrial good.\\
A single logistic group can store 12 SP and has 3 Minima production slots if it assumes the Production stance. A Goliath-class ship adds 6 storage capacity and 2 Vexillus slots on top of that.
\begin{itemize}
	\item \textbf{Astartes Chapter Fleet:} Start with one logistic group. Fleet based Chapters have access to at least one huge and ancient battle barge, which has considerable production abilities in addition to their fighting power and enormous resilience. A battle barge has the following profile: 1$\|$2$\|$1 (up to 8 SP per round). \\
	Between the battle barge and the logistic group, a total of 30 SP can be stored and 11 SP produced per round.
	
	\item \textbf{Ad-Mech Explorator Fleet:} Start with two logistic groups and an Ark Mechanicus, which has the following profile: 2$\|$2$\|$2 (up to 12 SP per round).\\
	This makes a total storage capacity of 36SP and a production rate of 18 SP per round.

	\item \textbf{Imperial Guard Task Force:} Start with a single logistic group and one Goliath-class factory ship for a total storage of 20 SP and a production rate of 7 Sp per round.
\end{itemize}

\subsection{Procuring Supplies}
Beside the production abilities of their logistic groups, Battle Groups can restore their supplies in the following ways.
\begin{itemize}
	\item \textbf{Faction Holds:} If the Battle Group spend at least one round within a system containing one of their faction holds, it can transfer an arbitrary amount of SPs from planet to Battle Group (and vice versa).
	\item \textbf{Imperial Subvention:} Employing tactful negotiation (or using dire threats) it is usually possible to draw some supplies from any loyal Imperial planet the Battle Group is currently visiting. The amount it will get is highly situational though and sometimes planets can really not spare anything (e.g. if it currently engaged in war itself or is still recovering from one).
	\item \textbf{Spoils of War:} A tradition as old as mankind: the winner gets it all. Scavenging the fields of battle (at least one turn) can upturn many salvageable supplies.
	\item \textbf{Supply Line:} Notably employed to keep massive, prolonged battle zones supplied, supply lines are a temporarily trade route between a faction hold and a warzone, shipping an arbitrary number of SPs from the hold to the warzone akin to the [Faction Hold] option. To set up a supply line, one can either delegate own void ships to ferry to and fro or request such services from another factions (e.g. Rogue Traders or merchant houses). Setting up a supply line takes at least as many rounds as a warp travel would take from the faction hold to the warzone plus two additional rounds if third party ships are used.
	\item \textbf{Rewards:} Other factions might provide material support for provided services, usually as part of a contract. A planetary governor might be willing to share some of his households funds with a Battle Group in exchange for a public parade in his capital to lessen the fears of his population or a besieged system is paying anything they have left in the planetary silos to their saviors.
	\item \textbf{Trade:} Trading some of the Battle Groups assets for much needed supplies is usually considered a last resort by Commanders but might be able to save a crusade from certain failure. A Commander might be willing to provide the service of one of its regiments to a Rogue Trader for an extended time period, in exchange for a substantial amount of supplies. 
\end{itemize}

\section{Commanders \& Heroes}
A commander is an accomplished leader of a faction who is put in command of a Battle Group. The faction leader of each faction is the overall commander and usually the commander of the key forces of the faction and is considered a Hero as well (see below).
Secondary Commanders can be assigned to new Battle Groups as necessary. Commanders can gain traits and followers (advisors, bodyguards, ...) during the course of their career.

Heroes are singular entities renowned throughout their factions and possibly beyond. Heroes can move around on their own - they do not need a commander or large fleet. A single small warp-capable ship will suffice.
Heroes can be linked to a Battle Group, providing a morale boost to all units therein. 
Whether or not they are linked to a Battle Group, heroes have different skills they can use to influence the world around them.
One hero can only use one active skill per round.
Similar to Commanders, heroes will acquire traits and followers throughout their career.

\subsection{Skills}
Skills appear in two variants: General (passive) and Unique (active) skills.
\begin{itemize}
	\item \textbf{General:} Skills in this category are passive in nature and (usually) revolve around strengthening own units, speeding up (re-)deployment or reducing travel time. Secondary Commanders and heroes start with one randomly determined General skill. Faction leaders start with two. The following list offers a selection of the most common General skills available, but the GM and players are encouraged to come up with additional ones.
	\begin{itemize}
		\item \textbf{Command Authority:} Units of the associated Battle Group have a slightly increased morale (10\%) and take no CE penalty when cooperating with allied forces in the same battlefield.
		\item \textbf{Cooperative Forces:} The Battle Group has an increased CE when fielded alongside allied units.
		\item \textbf{Rapid Relocation:} Efficient protocols reduce the (re-)deployment delay of the Battle Group by one round.
		\item \textbf{Omnipresent Logistics:} Battle Group units have a 10\% increased recovery rate.
		\item \textbf{Crash Course Recruitment:} Unit recruitment in the local system is reduced by 25\% to a minimum of one round.
	\end{itemize}
	\item \textbf{Unique:} Skill of this group are unique to a character, allowing him to influence his environment in a specific and active manner. Secondary Commanders and Heroes start with one such skill, while faction leaders start with two. The following are examples to spark the creativity of GM and players alike.
	\begin{itemize}
		\item \textbf{Pulled from the Brink:} The character is determined to not let any void ship go to waste even if others consider it beyond salvation. He can spend a full round to analyze the hulks of a void battle in the order to seek out salvageable wreckage. Such ships are considered at 0\% HP and can be repaired according to the normal rules (see section \ref{recovery_section}). He has also a chance to find usable ship parts when doing so.
		\item \textbf{Screening Protocol:} The Apothecari routinely screens the population of visited worlds to find suitable candidates for Astartes initiation. When performing this action for one or more rounds, he has a chance to detect genestealer infestations, mutations and diseases in the population in addition to a chance of finding new Neophytes for his chapter.
		\item \textbf{Artificer:} Given enough time and access to resources, the Master of the Forge (Tech Marine Hero) is able to craft even the most advanced war-gear of the chapter, including Terminator armor (\textit{Part, Chapter Hold, 8 rounds}) and Dreadnought chassis (\textit{Chapter Hold, 16 rounds}). This action requires the Battle Group or chapter hold to use the Produce stance (see section \ref{stances}) but does not count against its production slots.
		\item \textbf{Purge:} The Commissar is signing a Battle Group or even system wide order to purge the unclean. Any unit of the Battle Group that has one of the following traits will suffer HP damage and has a high chance of loosing the following hidden traits: Infected, Tainted, Corrupted and Possessed. A system wide order will inflict HP damage to a planet and has a chance to fully remove the trait from the local population.
	\end{itemize}
\end{itemize}

\subsection{Traits \& Followers}
During their career, commanders and heroes can accumulate character traits through special actions. 
Traits are usually passive in nature and provide a benefit in a very specific situation. 
Followers on the other hand represent the exalted personnel making up a characters counselorium. 
Most followers increase the efficiency of the character in handling a certain unit type or better the cooperation with other factions. 
New followers are usually gained through contracts or critical faction events. 
A character can have an unlimited number of traits but only ever have six followers in his staff. 
Should a character have more than six followers, he can change the composition of his counselorium once per year. 
Commanders and heroes start the game with a single trait and no followers.
The side bars below give examples for traits and a followers.
\begin{DndSidebar}{Siege Expert [Trait]}
Through fighting years upon years of bloody sieges, the character has developed a high mastery in this particular area of warfare.
Units of his Battle Group suffer 5\% less HP and Morale damage from ongoing sieges, while the enemy suffers an additional 5\% Morale penalty per round he is besieged.
\end{DndSidebar}
\begin{DndSidebar}{Know the Path [Trait]}
The character may see patterns and causations in navigational data that others do not. This allows him to plot safe paths through normally impassable or hazardous areas. 
\end{DndSidebar}
\begin{DndSidebar}{Navy Attachée [Follower]}
The character has won the loyalty and duty of a long-term Navy officer for his staff. His experience and connections increase the reputation with any Navy characters by 20. It does also boosts the Battle Groups CE considerably, if Navy assets are available in orbit over a contested world.
\end{DndSidebar}
\begin{DndSidebar}{Experienced Navigator [Follower]}
This particular experienced Navigator of a high-standing family is well able to steer the ships of the characters Battle Group through dangerous passages and even dares to steer the ships close to (or even into) warp storms or the shadow of the warp - although the latter is far from recommended.
\end{DndSidebar}


\chapter{Combat}
\section{Stances} \label{stances}
The upkeep cost of an entity is decided by the Stance(s) it is assuming during a round. Five types of Stances are defined here but some factions or Commanders may get access to additional ones - or loose access to some of those defined below. \\
Some Stances can be combined. In such a case, add together the upkeep cost of each activity to get the total upkeep. If two Stances reduce Warp travel, that penalty does not stack.

\subsection{Idle}\label{idle_stance}
This is the most basic "activity" possible and is generally frowned upon. This is only acceptable during times of severe resource shortage. Entities in this stance try to keep their upkeep as low as possible. Units are confined to their base without field training, planets stop their industry and send their workers home and fleets are put into high anchor or a stable orbit.
	\begin{itemize}
		\item \textbf{Cost:} 0 SP.
		\item Planets and other producers loose their basic production and halt any projects and recruitments.
		\item Units loose 1 experience points per round while staying idle.
		\item This activity cannot be combined with other activities.
	\end{itemize}

\subsection{Move}\label{move_stance}
Moving a single star ship through the void is a costly endeavor, but moving an entire fleet with accompanying ground forces is a herculean effort.
\begin{itemize}
	\item \textbf{Cost:} 1 SP.
	\item Unlocks inter-system warp travel and large scale intra-system ground force redeployment.
	\item This can be combined with: Produce, Brace and Recover. If combined with Produce, reduce the number of hexes you can travel by half (round down).
\end{itemize}

\subsection{Produce}\label{produce_stance}
This activity represents the expenditure of raw material to produce something (usually supplies, a unit, or a planetary advancement). See table \ref{production_table_units} for example unit production projects.	
\begin{itemize}
	\item \textbf{Cost:} 2 SP.
	\item Unlocks the production slots of the entity (if any).
	\item Unlocks the basic production of the entity (if any).
\end{itemize}

\subsection{Brace}\label{brace_stance}
Being prepared for war can be a matter of life and death in the endless wars of the 42nd millennium. Units have to perform regular live-fire drills and regular combat exercises to keep their edge. Planetary defense forces need regular drills as well. Walls and gun emplacements need to be repaired and calibrated. Fleets have to patrol and perform maneuvers. Orbital defenses require regular maintenance and calibration just like ground emplacements.\\
Light skirmishes with enemy forces are covered in this stance as well.
\begin{itemize}
		\item \textbf{Cost:} 2 SP.
		\item Units gains one experience point per turn
		\item Non-Unit entities (e.g. planets) can raise their HP beyond 100\% by taking this action. It rises by 10\% per round to a maximum of 150\%.
		\item Units taking this activity gain the 'on guard' trait for the duration of the activity thus raising the chance to detect enemy activity in the local area.
		\item Can be combined with the Move, Recover and Produce activities. 
		\item If combining this stance with Produce, non-unit entities reduce their base production and available production slots by half (round up).
\end{itemize}

\subsection{Fight}\label{fight_stance}
In the far future, there is only war. While not entirely accurate for every world in the Imperium, it certainly is for the Imperium at large. War is an unmatched devourer of resources such as food, medical supplies, ammunition, soldiers, war-gear; all are spent in vast quantities every day to keep the Imperium intact for just one more day. 
This activity represents a major military conflict, leaving no time for anything else besides. Minor military activity such as guard duty, light skirmishes, or patrolling fall under Brace.
\begin{itemize}
	\item \textbf{Cost:} 3 SP + 1 per 10 units (starting with the 11th unit).
	\item Units gain experience from combat. This can range from 2 to 8 experience points depending on the scale and length of the fighting as well as the outcome - victory yields more than defeat.
	\item Units cannot combine this stance with any other.
	\item Non-Unit entities can combine this stance with any other as long as its HP remains above 50\% with the following constraints: 
	\begin{itemize}
		\item Does no longer benefit from trade routes.
		\item Production slots and base production are halved (round down)
		\item Any recruitment process has a 33\% chance to not make any process during any given round due to enemy intervention.
		\item HP increase from the Brace and Recover stance are reduced by half (round down).
	\end{itemize}
\end{itemize}

\subsection{Recover}\label{recover_stance}
Combat losses and collateral damage are inevitable in the struggle for mankind's survival. Restoring good order, repairing damaged war-gear, and filling up the losses with new recruits is the purview of the Recover activity. The resource demands between two given units can be very large, thus the abstract notion of recovery points (RP) is used. The higher the RP of a unit type, the higher the cost to recover it (see \ref{recovery_table} for details).
\begin{itemize}
	\item \textbf{Cost:} 1 SP per five recovery points used per round (round up).
	\item Allows the entity in question to restore HP. Note that advanced units may have additional requirements that must be fulfilled before recovery kicks in (e.g. some may require a certain tech-level to be met).
	\item Recovering causes the unit to loose some experience points as new recruits fill up the ranks of dead veterans. For mechanical units, this represents the introduction of new parts (and thus machine spirits) into a greater whole, introducing minute changes in its behavior. A unit looses one experience point for each 20\% HP it recovers. Experience will never drop to negatives.
	\item Non-Unit entities recover 10\% per turn.
	\item Can be combined with the Move, Brace and Produce activities.
\end{itemize} 

\section{Combat Resolution}


\chapter{Player Factions}
The following sections provide faction specific rules, goals and information.

\section{Adeptus Astartes}
The Space Marines are the pinnacle of warriors within all the arsenal of the Imperium of Men and are creations of the Emperor himself.

\subsection{Geneseed}
Each warrior of the Adeptus Astartes carries within him two progenoid glands - an organ containing the DNA required to reproduce the unique organs which are fundamental in the creation of Astartes creation. 
The first gland is ready for harvest once the recruit reaches the rank of Devastor, the second is ready for harvest after his ascension to Tactical rank and only ever collected after the death of the marine.\\
Because of the fundamental requirement for Geneseed in the recruitment process, it is the single most valuable resource of a chapter and is thus tracked in-game as well.

\subsection{Recruitment Process}
The creation of an Astartes warrior is neither simple nor fast. At the start of the process, suitable candidates are required. Each chapter has its own, millenia old recruitment criteria and practices to select suitable youths. 
In the game, chapter holds will irregularly produce one \textit{batch} of Aspirants ready for implantation.
The batch has to be brought to the Chapters Fortress Monastery (or its flagship in case of fleet-based chapters) to start the implantation - which requires 12 Geneseed.
The implantation is a production process takes two rounds and is an Exactus-rated project and produces one squad of Neophytes.
Neophytes are first trained as Space Marine Scouts, which takes two rounds (see \ref{production_table_units}) at which point they can be fielded as a unit alongside their full fledged Battle Brothers.\\

\subsection{Astartes Units and Experience}
Being the pinnacle fighting force of the Imperium, Astartes quickly accumulate vast amounts of combat experience. 
In turn the requirements to rise in rank are also much higher than in any other fighting force. 
Astartes units use the following experience steps to determine their rank. 
Note that the rank of a squad limits which battle field roles it may assume. A unit may always be deployed in a 'lower' role but never as a higher one.
Once a squad reaches a new rank, its XP is reset to 0 for easier bookkeeping.
Astartes squads can not reach a new rank by training alone but only by means of smiting the foes of the Emperor.
\begin{itemize}
	\item \textbf{Scout:} The first rank after introduction into the chapter. Scout units are not yet ready to wear full battle plate as the black carapace is not yet fully matured and ready to interface with the armor.
	\begin{itemize}
		\item Combat Roles: Scout, Bikers
		\item XP: 0
		\item Max Morale: 100\%
	\end{itemize}
	\item \textbf{Devastor:} The second rank sees the marines deployed in a heavy fiore support role, allowing them to get used to fight in Servor Armor and wield the weapons of a full battle brother and the heavy weapons found within the arsenal of the chapter. 
	\begin{itemize}
		\item Combat Roles: Devastor, Hellblaster, Aggressors, Supressor, Eliminator, Infiltrator, Centurio
		\item XP: 12
		\item Max Morale: 130\%
	\end{itemize}
	\item \textbf{Assault:} The third rank sees the marines deployed where the fighting is thickest and up close and personnel. Equipped with jump packs and close-range weaponry, these squads are used as the mobile quick-response force, taking the fight to the enemy and support pressed heavily battle brothers. 
	\begin{itemize}
		\item Combat Roles: Assault, Inceptors, Reivers
		\item XP: 16
		\item Max Morale: 150\%
	\end{itemize}
	\item \textbf{Tactical:} Central rank of a chapter, tactical marines are the allrounder unit, capable to deal with nearly any situation, tactic and weapon of the Adeptus Astartes.
	\begin{itemize}
		\item Combat Roles: Tactical, Intercessor, Incursors
		\item XP: 20
		\item Max Morale: 200\%
	\end{itemize}
	\item \textbf{Veteran:} The final rank an Astartes unit reaches sees them reach Veteran status, opening the final and most sacred chambers within the armories of the chapter. 
	\begin{itemize}
		\item Combat Roles: Veteran, Terminator
		\item XP: 40
		\item Max Morale: 250\%
		\item Special: allows recruiting of a new hero or commander upon reaching this rank
		\item Sepcial: if this unit is destroyed (but recovered) a single dreadnought can be recruited
	\end{itemize}
\end{itemize}



\section{Adeptus Mechanicus}

\section{Imperial Guard}

\onecolumn
\chapter{Tables}
\begin{center}
 \begin{longtable}{l l l l l}\toprule
	Unit Name & HP & Requirement & RP & Rate \\ \endhead\midrule
	Regular Infantry & any & Minima & 0.5 & 40\% \\
	Special Infantry & any & Minima & 1 & 33\% \\
	Mechanized & any & Vexillus & 1 & 33\% \\
	Armored & any & Vexillus & 2 & 25\% \\
	\makecell[cl]{Astartes} & $\ge$75\% & - & 0.5 & 5\%\\
	\makecell[cl]{Astartes} & $<$75\% & - & - & 0\%\\  
	\makecell[cl]{Astartes} & $\ge$50\% & Hold or Fleet & 1 & 40\%\\
	\makecell[cl]{Astartes} & $<$50\% & Hold or Fleet & - & special\footnote{Needs recruits from the next lower experience level.}\\
	Knight & $\ge$50\% & \makecell[lc]{Vexillus or\\Knight World} & 1 & 25\%\\
	Knight & $<$50\% & \makecell[lc]{Exactus} & 2 & 20\%\\
	Titan & $\ge$75\% & Vexillus & 3 & 15\% \\
	Titan & $\ge$50\% & Exactus & 4 & 15\% \\
	Titan & $<$50\% & Exactus & 4\footnote{Plus one or more randomly-determined parts depending on the severity of damage suffered.} & 15\% \\
	Void Ship & $\ge$75\% & - & 2 & 5\%\\
	Void Ship & $\ge$75\% & Space Port & 2 & 25\%\\
	Void Ship & $\ge$50\% & \makecell[cl]{Vexillus,\\Space Port} & 4 & 20\%\\
	Void Ship & $\ge$25\% & \makecell[cl]{Vexillus,\\Space Port} & 4\footnote{Plus one randomly-determined part} & 10\%\\
	Void Ship & $<$25\% & \makecell[cl]{Exactus} & 5\footnote{Plus two or more randomly-determined parts}  & 10\%\\
	\bottomrule
	\caption{Recovery table by unit type.\label{recovery_table}}\\
\end{longtable}	
\end{center}

\begin{center}
\begin{longtable}{l l c c c } \toprule
    Name & Type & Tech Level & Time & Parts \\ \midrule\endhead
    \makecell[cl]{Basic Infantry} & Basic & Minima & 1 & - \\ \addlinespace
    \makecell[cl]{Mechanized Infantry} & Advanced & Vexillus & 3 & - \\ \addlinespace
    \makecell[cl]{Special Infantry} & Advanced & Vexillus & 3 & - \\ \addlinespace
    \makecell[cl]{Pulled Artillery} & Advanced & Vexillus & 3 & - \\ \addlinespace
    \makecell[cl]{Leman Russ Hull}  & Part & Vexillus & 2 & - \\ \addlinespace
    \makecell[cl]{Turret (Basic)} & Part & Vexillus & 1 & - \\ \addlinespace
    \makecell[cl]{Turret (Advanced)}  & Part & Exactus & 2 & - \\ \addlinespace
    \makecell[cl]{Leman Russ Assembly}  & Assembly & Vexillus & 1 & \makecell[cl]{Hull,\\Turret} \\\addlinespace
    \makecell[cl]{Astartes Scouts} & Advanced & Minima & 2 & Neophytes \\ \addlinespace
    \makecell[cl]{Astartes Wargear} & Part & Exactus & 4 & - \\ \addlinespace
    \makecell[cl]{Knight Chassis} & Part & Vexillus & 1 & - \\ \addlinespace
    \makecell[cl]{Knight Power Core} & Part & Exactus & 2 & - \\\addlinespace
    \makecell[cl]{Titan Basic Weapon} & Part & Vexillus & 3 & - \\ \addlinespace
    \makecell[cl]{Titan Advanced Weapon} & Part & Exactus & 3 & - \\ \addlinespace
    \makecell[cl]{Knight} & Assembly & Vexillus & \makecell{1 per\\part} & \makecell[cl]{Chassis,\\Core,\\Weapons} \\ \addlinespace
    \makecell[cl]{Warhound Chassis} & Part & Vexillus & 2 & - \\ \addlinespace
    \makecell[cl]{Warhound Power Core} & Part & Exactus & 4 & - \\\addlinespace
    \makecell[cl]{Warhound} & Assembly & Exactus & \makecell{1 per\\part} & \makecell[cl]{Chassis,\\Core,\\Weapons} \\ \addlinespace
    \makecell[cl]{Warlord Chassis} & Part & Vexillus & 4 & - \\ \addlinespace
    \makecell[cl]{Warlord Legs} & Part & Vexillus & 2 & - \\ \addlinespace
    \makecell[cl]{Warlord Head} & Part & Exactus & 4 & - \\ \addlinespace
    \makecell[cl]{Warlord Power Core} & Part & Exactus & 6 & - \\\addlinespace
    \makecell[cl]{Warlord} & Assembly & Exactus & \makecell{1 per\\part} & \makecell[cl]{Chassis,\\Core,\\Legs,\\Head,\\Weapons}\\ \addlinespace
    \makecell[cl]{Ship Hull Section} & Part & Vexillus & 4 & - \\ \addlinespace    
    \makecell[cl]{Ship Core} & Part & Exactus & 6 & - \\ \addlinespace  
    \makecell[cl]{Ship Bridge} & Part & Exactus & 4 & - \\ \addlinespace 
    \makecell[cl]{Ship Engines} & Part & Exactus & 4 & - \\ \addlinespace 
    \makecell[cl]{Macrocannon Battery} & Part & Vexillus & 2 & - \\ \addlinespace 
    \makecell[cl]{Lance Battery} & Part & Exactus & 3 & - \\ \addlinespace 
    \makecell[cl]{Torpedo Tubes} & Part & Vexillus & \makecell{1 per\\2 tubes} & - \\ \addlinespace 
    \makecell[cl]{Flight Deck} & Part & Vexillus & 6 & - \\ \addlinespace 
    Frigate & Assembly & Vexillus & \makecell{1 per\\part}& \makecell{Hull,\\Core\\Engine\\Weapons} \\\addlinespace
    Cruiser & Assembly & Exactus & \makecell{1 per\\part}& \makecell{2xHull,\\2xCore\\Engine\\Bridge\\Weapons} \\\addlinespace
    Battleship & Assembly & Exactus & \makecell{1 per\\part}& \makecell{4xHull,\\2xCore\\2xEngine\\Bridge\\Weapons} \\\addlinespace
     
        \bottomrule
        \caption{Unit production table.\label{production_table_units}}\\
\end{longtable}
\end{center}

\clearpage
\twocolumn