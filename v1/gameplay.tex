%!TEX root = ./M42_Strategy.tex
\chapter{Gameplay}
\section{Introduction}
This documents attempts to provide a collection of basic rules enabling players and GMs to play a strategic wargame set in the grim-dark future of the 42nd millenia and explore the depth and scale of the warhammer universe. Players will take on the mantle of a powerful faction leader of the Imperium of Men and seek to defend and where possible expand their factions influence, power and area of control by any means necessary.\\
The game system does not go into detail about the themes, unit types, sub-factions or nomenclature of the warhammer universe, assuming interested players are familiar with most, filling the remainder as required by reading up on any of the fan-written wiki resources\footnote{e.g. \url{https://warhammer40k.fandom.com/wiki/Warhammer_40k_Wiki}}.\\
The game system is taking heavy inspiration from the Total War game series, especially the Warhammer Fantasy installations, combined with the hex-based map known from the civilizations game series. It is far from complete at this stage and there are many more subsystems planned for the future, including a more differentiated resource and supply system, planet advancements and character development for commanders and heroes.

\section{Sector Map}
The game uses a map with a hex grid, with each grid representing a several cubic light year large portion of space. 
Imperial and known enemy systems are shown on the map but it is important to keep in mind that vast volumes of space between these known systems are partly or even completely unknown, so an empty grid is by no means devoid of any star systems. Even in a future where there is only war, human-inhabited planets are rediscovered even within well-charted and traveled sectors of the Imperium - not to mention systems infested with terrifying Xenos.\\
In addition to the known systems, the positions of each players commanders are denoted as well as known (or suspected) positions of large enemy battle groups.

\section{Events}
Events are the prime tool for the GM to present the players with a dynamic, ever-changing galaxy. Events have a couple of properties which are explained in the following subsections.
For an example event layout, see the sidebar below.

\subsection{Origin}
The origin of the message or the location of the event.

\subsection{Severity}
The events are pre-sorted and categorized by aides into the following three severity categories:
\begin{itemize}
	\item \textbf{Minoris:} Considered unimportant by most of the aides, someone decided to relay the information to their superiors anyway. Most events in this category are used to create the impression of a living galaxy. However they are an excellent tool to foreshadow coming changes or bigger events and as always, the severity of an event changes rapidly with perspective. Minor events have only a minor gameplay effect (e.g. prolonging a recruitment process by one round, expenditure of an additional SP during upkeep phase) if they have one at all.\\
	Examples: sabotage in a local manufactoria, small civil unrest in a Hive, spreading non-lethal illness in ones troops, intercepted heavily encrypted vox traffic of unknown origin, unexpected warp-travel signatures approaching the system and so on.
	\item \textbf{Capitalis:} Important events, noteworthy of the attention of the faction leader. Events in the Capitalis category will most likely impact the strategic considerations of the players. They have a notable gameplay impact and/or limit the available options.\\
	Examples: civil war on a planet, damaged warp drives, attack on an important supply depot, uncovered activity of a villain (enemy hero), intelligence on enemy troop movement.
	\item \textbf{Critical:} Faction-specific, highly important event impacting a faction asset, faction leader or an event of sector-wide impact.\\
	Examples: an attack or lost communication with a faction hold, large Ork Waaagh! or Hive Fleet on the move.
\end{itemize}

\subsection{Effect}
The direct gameplay effect of the event - if any. 
Common effects include: change in warp travel time, change in production time, change in unit morale or hp, modified supply situation, ...

\subsection{Duration}
Many events are snapshots of a situation and do not have a noteworthy duration. Snapshot events are denoted by a simple dash in the duration property. Other events may have a fixed duration (given in rounds) or are simply considered ongoing if no exact duration can be given.

\subsection{Reliability}
Information in the Imperium are transported via astropathic link. Longer range communication is usually taking several hops, that is they are transmitted, transcribed then retransmitted  by another astropathic quire. Thus the number of hops can be used as a metric for the reliability of the transmitted information. A zero-hop information (direct link) is usually considered very precise while the same message after five or more hops is considered all but useless by most.
\begin{DndSidebar}{WAAAGH! Snaggatooth Moves!}
\begin{itemize}
	\item \textbf{Origin:} Sub-Sector Magnifica Imperialis. Archbishop Pontius Crucius.
	\item \textbf{Severity:} Capitalis
	\item \textbf{Effect:} -
	\item \textbf{Duration:} Unknown.
	\item \textbf{Reliability:} 1
	\item \textbf{Brief:} First elements of WAAAGH! Snaggatooth have been sighted within Sub-Sector Magnifica Imperialis, performing destructive raids of outlying systems. Large fleet movements detected by local astropaths, indicating the long expected assault is finally on the way towards the main system.

\end{itemize}
\end{DndSidebar}

\section{Requests}
Requests are a special type of events, representing the request of a influential group or individual towards one or more players. Requests have the same properties as events but also add the following information:

\subsection{Contractor}
The person or faction to propose the request as well as some background information of the person or institution in question - if available.

\subsection{Contract}
Brief summary of what is asked and what is offered in return (if any).

\begin{DndSidebar}{Request for Astartes Force}
\begin{itemize}
	\item \textbf{Contractor:} Magos-Explorator Zizar Rex (Adeptus Mechanicus)
	\item \textbf{Contract:} In exchange for a Tactical Squad of Astartes to accompany him on a dangerous excavation site within non-imperial territory, the Magos offers an entire shipping of ammunition, ranging from bolter shells to void torpedoes as well as a share of the spoils from his excavation endeavor.
	\item \textbf{Decision Due:} 1 round
\end{itemize}
\end{DndSidebar}


\section{Gameplay Phases}
During a round, there are four phases.
\begin{itemize}
	\item \textbf{Event-Phase:} Right at the beginning of a turn, a list of new and ongoing events is listed by the GM.
	\item \textbf{Action-Phase:} In this phase, the players plan their actions and commit to them. Stances for all armies are selected and heroes may use their active skills. Production plans are communicated and units recruited. Diplomacy happens.
	\item \textbf{Resolution-Phase:} Wars are fought, lost and won in this phase, be they be chosen by the players or triggered by the foe. Fate dice are tossed in this round.
	\item \textbf{Upkeep-Phase:} At the end of a round, bookkeeping is done. Experience is tracked and noted down. Supply situation is updated. Progress of any ongoing production, recovery or recruitment process is tracked.
\end{itemize}