\section{Supply, Upkeep and Activity}
Any single entity in the game, from a single unit to vast armies and entire planets, is storing, consuming and or producing certain amounts of resources to oil the complex machinery that is the Imperium of Men.

\subsection{Upkeep}
Be it a single unit or an entire planet, most entities are not entirely self-sufficient in the grim-darkness of the 42nd millennia. This is represented by the upkeep of the entity. Each round, the listed amount of resources is consumed.


\subsection{Activity}
The exact upkeep cost of an entity is decided by the activity it is taking during a round. Four types of activities are discerned. Some of those can be combined. In such a case, add together the upkeep cost of each activity to get the total upkeep.

\begin{itemize}
    \item \textbf{Idle:} This is the most basic activity and generally frowned upon. Only acceptable during times of severe resource shortage, entities in this stance try to keep their upkeep as low as possible. Units are confined to their base without field training, planets stop their industry and send their workers home and fleets are put into high anchor or a stable orbit. Effects:
    \begin{itemize}
        \item Planets loose their basic production and halt any projects and recruitment processes.
        \item Units loose 1 experience point per round while staying idle.
        \item The listed upkeep cost of the Idle activity has to be paid every round, regardless of other activities.
        \item This activity cannot be combined with other activities.
    \end{itemize}
    
    \item \textbf{Produce:} This activity represents the expenditure of resources to produce something (usually another resource, a unit or planetary advancement). Mostly used by planets and similar fix entities, but armies too have some limited production options like the set-up of a fortified position or a supply depot.
    Effects:
    \begin{itemize}
        \item Enables the production slots specified for the entity.
        \item The entity crafts its specified basic production each round this activity is active.
        \item Can be combined with the Brace and Recovery activities.
    \end{itemize}
    
    \item \textbf{Brace:} Being prepared for war is a matter of life and death in the endless wars of the 42nd millennia. Units have to perform regular life-fire drills and regular exercise to keep their edge. Planetary defense force need regular drills as well, walls and gun emplacements need to be repaired, fleets have patrol and perform maneuvers and orbital defences require regular maintenance. Effects:
    \begin{itemize}
        \item Units gains one experience point per turn
        \item Non-Unit entities (e.g. planets) can raise their CE beyond 100\% by taking this action. It rises by 10\% per round to a maximum of 150\%.
        \item Units taking this activity gain the 'on guard' trait for the duration of the activity thus raising the chance to detect enemy activity in the local area.
        \item Can be combined with the Recovery and Produce activities. If combined with the Production activity, the entity looses half its production slots of the Minima and Vexillus level (round down).
    \end{itemize}
    
    \item \textbf{Fight:} In the far future, there is only war. While not entirely accurate for every world in the Imperium, it certainly is for the Imperium at large. War is an unmatched devourer of resources: food, medical supplies, ammunition, soldiers, war-gear: all are spent in vast quantities every day to keep the Imperium intact throughout another day. This activity represents a major military conflict, leaving no time for anything else beside. Minor military activity, e.g. guard duty, light skirmishes or patrolling fall under 'Brace'.
    Effects:
    \begin{itemize}
        \item Units gain experience from combat (2-8 experience points depending on the scale of the fighting and outcome - victory yields more than defeat).
        \item Cannot be combined with any other activity.
    \end{itemize}
    
    \item \textbf{Recover:} Combat losses and collateral damage are inevitable in the struggle for mankind's survival. Restoring good order, repairing damaged war-gear and filling up the losses with new recruits is the purview of the recover activity.
    Effects:
    \begin{itemize}
        \item Allows the entity in question to restore CE. Note that advanced units may have additional requirements that must be fulfilled before recovery kicks in (e.g. some may require a certain tech-level to be met).
        \item Recovering causes the unit to loose some experience points as new recruits fill up the ranks of dead veterans. For mechanical units, this represents the introduction of new parts (and thus machine spirits) into a greater whole, introducing minute changes in its behavior. A unit looses one experience for each 20\% CE it recovers.
        \item Can be combined with the Brace and Produce activities.
    \end{itemize}   
\end{itemize}

\subsection{Supply Buffer}
Every entity has a so-called supply buffer, representing the amount of supplies it currently has stored in a way to easily access it if need be - e.g. to its own upkeep. Unless otherwise noted, an entities upkeep is taken from its supply buffer at the end of each round, reducing the remaining amount according to the entities current upkeep. Each resource is tracked individually and the maximum size of each buffer might vary between resources. In the 'Buffer' section of an entities description, the individual buffer sizes are given like this: GG(4/8), meaning that the entities buffer of GG is currently half empty. It has 4 units of GG left from a maximum buffer capacity of 8. 

\subsection{Resupply} \label{resupply}
There are many ways to resupply - that is to raise the supply counter of an entity. The following are the basic possibilities that are open to most factions.
\begin{itemize}
    \item \textbf{Trade:} Old as humankind, trading is most basic way to get what you need - although usually at a price. Entities in the same system can simply exchange goods (from their respective supply buffers) if both sides agree to the exchange - which is far from given, even if both sides are loyal imperial organisations. Trading between systems is another matter entirely, requiring intergalactic travel, void-capable merchant ships (and a navigator), trading allowances, certificates and lots (and lots and lots then some) triplets. See section \ref{trade_routes} for trade routes.
    
    
    \item \textbf{War Spoils:} Since the beginning of time, crusading armies have put any leftovers of enemy equipment to good use to keep the own soldiers supplied. Imperial armies are not above this should the need arise. Destroyed vehicles can be cannibalized for ore and rare minerals required to repair the own battle gear and vehicles. Captured soldiers can be mind scrubbed and put to work as servitors and even send to the front as penal legionaries. Lastly, Bolt shells, grenades, fuel and rations will serve anyone who possesses them after all.
    
\end{itemize}


\subsection{Tithes}
\todo{To be added later...}