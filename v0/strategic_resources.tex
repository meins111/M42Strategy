\section{Strategic Resources} 

 Resources are produced, consumed and wasted in vast amounts by the endless wars of the far future. Planets produce and sometimes consume a certain amount of resources, depending on the planet type. E.g. a Hive world is able to produce a decent amount of goods as well as a steady supply of recruits for the armies of the Emperor – but only if supplied with food. Diametrically opposed to that, Agri-Worlds produce enormous amounts of food, but have only minimal defenses and rely on regular import of promethium and tools to keep up production. \\
\todo{Maybe talk about resource units and consumption here.}
\subsection{Resource Types} 

The following resources are tracked in the game: 
\begin{itemize} 
    \item \textbf{General Goods (GG):} Catch-all term for uniforms, basic weapons and ammunition, tools, medical supplies, power packs, etc. Required by nearly any unit during production as well as supply requirement. 
    \item \textbf{Food:} Man must eat (and drink). What they consume is rather secondary but without food and water no war will be fought and ancient dynasties crumble into anarchy. Most units require this resource to keep supplied. 
    \item \textbf{Men:} If there is one thing the Imperium of Man has in abundance, then it is unlimited quantities of manpower. Many military assets consume one unit of this particular resource upon creation. 
    \item \textbf{Ore:} Catch-em-all for any one of a stunning amount of raw materials used by the forges of the Imperium to build their weapons. Required to build anything, from bolters to prefabricated building parts and star ship hulls. All mechanical or mechanized units require this resource during production. 
    \item \textbf{Rare Minerals:} Catchphrase for high quality or particularly rare mining resources, required for producing advanced technology like warp drives, plasma weapons, engines, psi-technology or void shields. Required by most high-tech units during production. 
    \item \textbf{Promethium:} The Imperium of Man runs on promethium. Any vehicle, from civil ground cars to void ships use it as fuel. Flamers utilize it as a weapon against the Heretic and the Alien. Complex chemical products are based on it. All mechanical and mechanized units require it for supply. 
    \item \textbf{Shells:} Large calibre-shells, fired by any solid-projectile weapon the size of a Leman Russ Battlecannon or larger requires this resource to stay supplied. Shells are produced by planets from Ore and Promethium. 
    \item \textbf{Missiles:} Fast, precise and powerful missiles are fielded by many units in the Imperial arsenal. As powerful an asset as they are, they need to be produced carefully, from skilled manufacturors or specialized and supervised machinery. Units that use missile weapons, require this resource to stay supplied. Missiles are produced by planets from GG, Ore and Promethium. 
\end{itemize} 

In addition to these, a couple of faction-specific resources are available: 
\begin{itemize} 
    \item \textbf{Neophytes:} Recruits of an Astartes Chapter -– the raw material from which squads of new Space Marines are forged. 
    \item \textbf{Gene-Seed:} The genetic material carried by all Space Marines. Used to allow new Neophytes to ascend to an Astartes. Collected from fallen Space Marines. Generated from Space Marines reaching Tactical-level.
    \item \textbf{Princeps:} Strong-minded, intelligent and well-trained individuals are required to pilot the knights and titans of the Adeptus Mechanicus. 
    \item \textbf{Officers:} Both, the Imperial Navy and Imperial Guard rely on the skill of its officers to win wars. Well trained individuals with command skill are a rarity under the 'protective' gauntlet of Imperial regime.
\end{itemize} 

\subsection{Trade Routes} \label{trade_routes}
Often, resources are produced far from the manufactoria that will use them to create war material. Thus, inter-system cargo trade routes are a matter of fact in the Imperium. To simplify the game experience, only trade routes crossing the sub-sector boundary will be tracked. Any cargo transfer within systems of the same sub-sector will be automatic. \\ 
\textbf{Note:} Should a system become contested with the enemy having void supremacy, its products cannot be transferred within the sub-sector either. \newline
\\
Trade routes usually follow similar paths as the Immaterium is far from being equally hard to traverse in all places. Thus, trade routes usually follow certain 'warp routes' - well-charted routes between sub-sectors. The entries and exits of these routes are usually guarded by the Imperial Navy to prevent pirates from attacking the merchant ships plying these trade routes. In times of immediate danger, however, these forces will have to be redeployed, leaving the trade lines vulnerable.  \\
A trade line is denoted as follows: Origin $\rightarrow$ Destination: Resource (Amount). E.g. Hastea $\rightarrow$ Atria: Food(4). 
In many cases, trade routes are chained, with several destination worked in sequence. This can be represented by several chained arrows, like this: Hastea $\rightarrow$ Atria: Food(4) $\rightarrow$ Meridian: Food (2)

\subsubsection{Establishing a Trade Route} 
To establish a new trade route (or supply line), proceed as follows: 
\begin{enumerate} 
\item Select the origin and target planet as well as the number (and type) of resources to transport. Obviously, only resources can be transported that are available on the origin planet. 
\item Select the route from source to destination. If 2/3 or more of the routes total path is following well-charted warp routes note down: 'well charted'. 
\item A merchant house will be selected (after the usual offer and bidding time), ships will be assembled and loaded. The delay between establishing a trade line and the arrival of the first wares at the target is as follows: \\ 
$0.5 \times \text{WareCount} + \text{TravelTime}$\\ 
The travel time is halved if the trade line has the 'well charted' property. See section \ref{warp_travel} for information about warp travel speed. 
\end{enumerate} 
An established trade line will remain active as long as it is not cancelled or interrupted. 
