
\section{Units, Recruitment and Recovery} 
\subsection{Unit Types}
The game uses the catch-all phrase unit to talk about a collection of (mostly) heterogeneous combatants. A unit range from a regiment of Imperial Guardsman, over a squad of Adeptus Astartes, to the God-Machines of the Adeptus Mechancius or a single void ship.\\
For the sake of recruiting, deployment and other generalized concepts, the following unit types are used:
\begin{itemize}
    \item \textbf{Basic Infantry}: A couple hundred men and women under arms without vehicle support of note or specialized training or equipment. Cheap and fast to recruit. Deployed en mass.
    \item \textbf{Advanced Infantry}: These units use higher grade equipment, training and often have vehicle support and thus have greater potential, but at the price of longer recruitment and the need for additional resources. Advanced Infantry is subdivided in mechanized and specialized infantry.
    \begin{itemize}
    \item \textbf{Mechanized Infantry:} An infantry unit with vehicle support, ranging from Recon-Units on Sentinel-Walkers, to towed artillery units and Iron Fist regiments, riding inside heavy Chimera APCs into the thickest of battle before deploying right into the enemies line.
    \item \textbf{Specialized Infantry:} Specialized infantry units, using special tactics, superior equipment or performing special tasks on the battlefield. Units falling into this category are: drop troopers, storm troopers, MASH unit.
    \end{itemize}
    \item \textbf{Armor} Heavy combat vehicles, like all Leman Russ pattern as well as self-propelled artillery pieces (Basilisk/Manticore).
    \item \textbf{Imperial Knight:} Combat walkers that are not yet counted as a true Titan, these machines are still able to wreak havoc in nearly any combat situation rivalling a super-heavy tank. A detachment of Knight (usually 4 walkers) can even threaten a Titan if they manage to close or surprise it.
    \item \textbf{Titan:} The god-machines of the Adeptus Mechanicus are the masters of the battlefield, only truly threatened by other Titan-grade units. It takes years to build even one of these mighty machines and they require specialized ships to carry, deploy and refit them. Titans are subdivided into scout and battle titans.
    \begin{itemize}
        \item \textbf{Scout Titans:} Represented by the Warhound class of the Adeptus Titanicus, these Titans (while considerably smaller than their greater cousins) still tower over small buildings and are able to topple over a battletank with a kick from its feet. Being much more agile and fast compared to a true battle titan, scout titans are used for reconnaissance, ambushes and flanking maneuvers. They are usually fielded in pairs of two.
        \item \textbf{Battle Titans:} Ranging from the old Reaver class, over the widespread Warlord class to the towering, ancient Emperor titan, battle titans represent the mechanical gods of war that flatten battle tanks with a mere step and lay waste to entire cities if let loose.
    \end{itemize}
    \item \textbf{Astartes Combat Squad:} A single squad of Space Marines are a devastating force, able to lay waste to hundreds of lesser enemies during a single mission. It takes years to train and "make" a Space Marine and every loss is felt dearly within the chapter.
\end{itemize}

\subsection{Macro Units}
Due to the sheer scale of combat in the grim darkness of the far future, a single unit usually does not suffice to turn the tide of war. Thus, the game uses the term Macro unit to refer to a collection of units of (potentially) different types that act as one game piece on the huge game board of the sector.  
While every faction has different names for its formations, all have three tiers of macro units, each one larger than the previous one. We will use the names of the Imperial Guard as generic name as they are quite intuitive. 
\begin{itemize} 
    \item \textbf{Battlegroup:} A small collection of (typically) up to five single units. The Navy splits this categiry into Patrols (2-6 ships) and Subsector Fleet (up to 15 ships), while the Adeptus Astartes call a formation of exactly five squads a Demi-Company and anything less than that a Combat Detachment. In the Imperial guard, the term Battlegroup is used for any formation consisting of anything between two and fifteen regiments. 
    \item \textbf{Army:} A large formation of units, usually twice to quadruple the size of a Battlegroup. The Navy equivalent is the Sector Battlefleet. In the Astartes terminology, a full company (including armor and air support) takes this position. 
    \item \textbf{Army Group:} Any formation larger than an army with no upper limit. If the Navy pulls all it Subsector fleets together to reinforce the sector battlefleet, they call it an armada. Only in the most dire situation will an Astartes Chapter go to war in chapter strength. The Adeptus Titanicus calls ist largest formation Legion. 
\end{itemize} 

\subsubsection{Astartes Chapter}
An Astartes Chapter is a highly self-sufficient organisation structure, with its own recruitment holds, void-ships, astropathic relays, navigator houses, manufactoria and thousands of chapter serfs. According the Codex Astartes, written by the Lord Regent Roboute Guilliman himself, a chapter consist of up to a thousand Adeptus Astartes, organized into 10 companies of 100 soldiers each, which in turn are sub-organized into two demi-companies of 5 squads each.\\
But even the most self-contained organization has need of resources being produced somewhere to fill up their stock from time to time. In addition, an Astartes Chapter is always at war readiness. That is, even if they currently are not engaged in active battle, the brothers are performing lengthy and tasking combat drills, which deplete the chapters resources nearly as fast as if they were waging actual war. To represent this, the entirety of the chapter has a supply counter for each resource it requires.
Planet-based chapters usually draw their required resources from their homeworlds (and/or additional honor-bound worlds). Fleet-based chapters on the other hand roam the galaxy and will often be far off systems incorporating chapter holds, who might supply them at short notice. 
Astartes chapters can resupply via one of the following means (in addition to the standard means listed in section \ref{resupply}):
\begin{itemize}
    \item \textbf{Honor-bound Systems:} Chapter holds as well as systems the chapter has gone to great length to protect, are considered honor-bound to the chapter and will go to great lengths to supply the chapter fleet with any resources they have available should the fleet pay a visit to their system.

    \item \textbf{Grateful Systems:} Systems that were recently (within one year) protected by the Angels of Death will usually agree to supply their saviours with whatever products they can. Should a system be saved multiple times (or in a particular spectacular and memorable way) the system may eventually become honor-bound.
    
    \item \textbf{Imperial Subvention:} By publicly declaring to be in dire need of supplies, an Astartes chapter may requisition resources from any loyal imperial system. While most regents are intelligent and pious enough to follow such a request, some may hold back on the amount of provided resources to keep their planets economy from stalling.
    Most chapters view this options as a last resort, to be taken only in the most grim of situation and it is generally considered a grave insult to the chapter honor to rely on external help in such a manner.
\end{itemize}


 \subsection{Supply Counter}
 All units have a supply counter representing the status of its tactical supplies - food, medical supplies, replacement parts, ammunition, and a myriad things more.
 Each round, in which the unit does not have access to all its required resources and is either in combat, training or redeployment, the supply counter is reduced by one. The Combat Efficiency of the unit is (heavily) influenced by the current value of its supply counter.
 \begin{itemize}
    \item \textbf{Counter $=$ max:} The unit is fully supplied. Combat Efficiency is slightly increased.
    \item \textbf{Counter $\ge$ 0:} The unit has enough tactical reserves to continue fighting without suffering major ill effect.
    \item \textbf{Counter $<$ 0:} The unit is suffering from supply shortages. Its Combat Efficiency is reduced - the severity depends on the unit type.
    \item \textbf{Counter $=$ -max:} The unit is no longer combat effective. The infantry or crew personal is either dead or starving. Vehicles and machines are no longer combat ready due to combat damage and lack of maintenance.
 \end{itemize}
 Consult the following table to check what kind of resources a unit in question requires to supply. Depending on the units main weaponry, consult section \ref{ammo} for details about the units ammunition requirement. If the general supply requirement and the ammunition requirement overlap, the unit only requires the greater number of the two. E.g. a unit of mechanized infantry in Hellhounds (main weaponry: heavy flamer) has general Promethium(1) upkeep and a weaponry-dependant upkeep of Promethium (2) thus, the unit requires two units of Promethium to restore one point of its supply counter.
 \begin{longtable}{l c l}\toprule
    Unit & \makecell{Supply\\Counter} & Resources\\\midrule \endhead
      Basic Infantry & 5 & Food, GG, ammo \\\addlinespace
      Specialized Infantry & 3 & Food, GG, ammo\\\addlinespace
      Mechanized Infantry & 4 & \makecell[lc]{Food, GG, Promethium, ammo}\\\addlinespace
      Armor & 3 & Food, GG, Promethium, ammo\\\addlinespace
      Knights & 2 & GG, Promethium, ammo\\\addlinespace
      Scout Titan & 3 & GG, Promethium, ammo\\\addlinespace
      Battle Titan & 2 & GG, Promethium, ammo\\\addlinespace
      \makecell[lc]{Astartes Combat\\Detachment} & 6 & \makecell[lc]{GG, Food,\\with Thunderhawk: Promethium}\\\addlinespace
      Astartes Chapter & 16 & \makecell[lc]{Food(2), Men(1), GG(2)\\Promethium(2), Ore, Rare Minerals\\Shells, Missiles}\\\addlinespace
      \makecell[lc]{Imperial Navy\\Recon Group} & 12 & Food, GG, Promethium\\\addlinespace 
      \makecell[lc]{Imperial Navy\\Subsector Fleet} & 8 & Food, GG(2), Promethium(2)\\\addlinespace 
      \makecell[lc]{Imperial Navy\\Sector Battlefleet} & 8 & Food(2), GG(3), Promethium(4)\\\addlinespace 
      
      \bottomrule
 \end{longtable}{}
 
 \subsubsection{Resupplying}
 To increase the supply counter, the unit must have access to all resource listed in the units supply section.
 A unit is considered to have access to a resource, if the system it is stationed in has access to the resource. Units in a warzone or in a system lacking the necessary resources, must be supplied via a Supply Line, shipping in the required resources from other worlds. Supply lines work exactly like Trade Routes (see \ref{trade_routes}). \\
 A unit can never have a supply counter greater than its maximum. The speed of resupplying depends on the unit size and the number of units in a system.
  \begin{itemize}
    \item \textbf{Single Unit or small Detachment:} Supply timer increases by six each round.
    \item \textbf{Single Macro Unit} Supply timer increases by four per round.
    \item \textbf{Single Battlegroup or up to four Macro Units} Increases by two per round.
    \item \textbf{Armies composed of more than four Macro Units} Require additional goods to be shipped in to supply them. Split the army into several Battlegroups of no more than four macro units. Each Battlegroup must have a separate supply line.
     
 \end{itemize}
 
 \subsection{Recruitment} 
To recruit a unit, the following prerequisites must be met: 
\begin{itemize} 
    \item The planet which shall recruit the unit must have a free production slot available, which is at the same level (or higher) than the unit it wants to recruit. 
    \item Any Planetary advancement requirements the unit may have has to be met. 
    \item Any one-time resource requirements must be met. This may happen at any round of the recruitment. 
    \item Any resource requirements must be met - at any round for the duration of the recruitment process. 
\end{itemize} 

\subsection{Recruitment Chart} \label{recruitment_chart}
\begin{center}
\begin{longtable}{l l c c c c c} \toprule
    Name & Type & Tech Level & Time & Parts & Resources & One-Time\\ \midrule\endhead
    \makecell[cl]{Basic \\Infantry} & Basic & Minima & 1 & - & Men, GG & - \\ \addlinespace
    \makecell[cl]{Mechanized\\Infantry} & Advanced & Vexillus & 3 & - & GG, Ore &  \makecell[cl]{Promethium,\\Men, Ammo} \\ \addlinespace
    \makecell[cl]{Special\\Infantry} & Advanced & Vexillus & 3 & - & GG & Men \\ \addlinespace
    \makecell[cl]{Pulled\\Artillery} & Advanced & Vexillus & 3 & - & GG, Ore & \makecell[cl]{Promethium,\\Men, Ammo} \\ \addlinespace
    \makecell[cl]{Leman Russ\\Hull}  & Part & Vexillus & 2 & - & GG, Ore & -\\ \addlinespace
    \makecell[cl]{Turret\\(Basic)} & Part & Vexillus & 2 & - & GG, Ore & - \\ \addlinespace
    \makecell[cl]{Turret\\(Advanced)}  & Part & Exactus & 3 & - & GG, Ore & Rare Minerals \\ \addlinespace
    \makecell[cl]{Leman Russ\\Assembly}  & Assembly & Vexillus & 1 & \makecell[cl]{Hull,\\Turret} & GG & \makecell[cl]{Promethium,\\Men, Ammo}\\\addlinespace
    \makecell[cl]{Astartes\\Scouts} & Advanced & Minima & 4 & - & GG, Food & \makecell[cl]{Rare Minerals,\\Neophytes,\\Geneseed(10)} \\ \addlinespace
    \makecell[cl]{Astartes\\Wargear} & Part & Exactus & 4 & - & \makecell[cc]{GG, Ore,\\Rare Minerals} & - \\ \addlinespace
    \makecell[cl]{Astartes \\ Squad} & Assembly & - & 1 & \makecell[cl]{Scouts,\\Wargear} & - & - \\ \addlinespace
    \makecell[cl]{Knight \\ Chassis} & Part & Vexillus & 1 & - & GG, Ore & Rare Minerals \\ \addlinespace
    \makecell[cl]{Knight \\ Power Core} & Part & Exactus & 2 & - & \makecell{GG, Ore,\\Rare Minerals} & Promethium \\\addlinespace
    \makecell[cl]{Titan\\Regular\\Weapon} & Part & Vexillus & 1 & - & GG, Ore & Ammo \\ \addlinespace
    \makecell[cl]{Titan\\Advanced\\Weapon} & Part & Exactus & 2 & - & \makecell{GG, Ore,\\Rare Minerals} & Ammo \\ \addlinespace
    \makecell[cl]{Knight} & Assembly & Vexillus & \makecell{1 per\\part} & \makecell[cl]{Chassis,\\Core,\\Weapons}  & GG & \makecell{Princeps,\\Promethium} \\ \addlinespace
    \makecell[cl]{Warhound\\Chassis} & Part & Vexillus & 4 & - & GG, Ore & Rare Minerals \\ \addlinespace
    \makecell[cl]{Warhound \\ Power Core} & Part & Exactus & 4 & - & \makecell{GG, Ore,\\Rare Minerals} & Promethium \\\addlinespace
    \makecell[cl]{Warhound} & Assembly & Exactus & \makecell{1 per\\part} & \makecell[cl]{Chassis,\\Core,\\Weapons}  & GG & \makecell{Princeps,\\Promethium} \\ \addlinespace
    \makecell[cl]{Warlord\\Chassis} & Part & Vexillus & 4 & - & GG, Ore & Rare Minerals \\ \addlinespace
    \makecell[cl]{Warlord\\Legs} & Part & Vexillus & 2 & - & GG, Ore & Rare Minerals \\ \addlinespace
    \makecell[cl]{Warlord\\Head} & Part & Exactus & 4 & - & \makecell{GG, Ore,\\ Rare Minerals} & - \\ \addlinespace
    \makecell[cl]{Warlord \\ Power Core} & Part & Exactus & 6 & - & \makecell{GG, Ore,\\Rare Minerals} & Promethium \\\addlinespace
    \makecell[cl]{Warlord} & Assembly & Exactus & \makecell{1 per\\part} & \makecell[cl]{Chassis,\\Core,\\Legs,\\Head,\\Weapons}  & GG & \makecell{Princeps,\\Promethium} \\ \addlinespace
    \makecell[cl]{Ship Hull\\Section} & Part & Vexillus & 4 & - & GG, Ore & - \\ \addlinespace    
    \makecell[cl]{Ship Core} & Part & Exactus & 6 & - & \makecell{GG, Ore,\\ Rare Minerals} & - \\ \addlinespace  
    \makecell[cl]{Ship\\Bridge} & Part & Exactus & 4 & - & \makecell{GG,\\Rare Minerals} & Ore \\ \addlinespace 
    \makecell[cl]{Ship\\Engines} & Part & Exactus & 4 & - & GG, Ore & - \\ \addlinespace 
     \makecell[cl]{Macrocannon\\Battery} & Part & Vexillus & 2 & - & GG, Ore & Shells \\ \addlinespace 
    \makecell[cl]{Lance\\Battery} & Part & Exactus & 3 & - & \makecell{GG, Ore, \\Rare Minerals} & - \\ \addlinespace 
    \makecell[cl]{Torpedo\\Tubes} & Part & Vexillus & \makecell{1 per\\2 tubes} & - & \makecell{GG, Ore,\\Missiles} & \makecell{Rare Minerals} \\ \addlinespace 
    \makecell[cl]{Flight\\Deck} & Part & Vexillus & 6 & - & GG, Ore & \makecell{Promethium,\\Missiles} \\ \addlinespace 
    Frigate & Assembly & Vexillus & \makecell{1 per\\part}& \makecell{Hull,\\Core\\Engine\\Weapons} & GG, Men & \makecell{Officers,\\Promethium} \\\addlinespace
    Cruiser & Assembly & Exactus & \makecell{1 per\\part}& \makecell{2xHull,\\2xCore\\Engine\\Bridge\\Weapons} & GG, Men & \makecell{Officers(2),\\Promethium} \\\addlinespace
    Battleship & Assembly & Exactus & \makecell{1 per\\part}& \makecell{4xHull,\\2xCore\\2xEngine\\Bridge\\Weapons} & GG, Men & \makecell{Officers(3),\\Promethium} \\\addlinespace
     
        \bottomrule
\end{longtable}
\end{center}
\subsubsection{Ammunition}\label{ammo}
Many units in the above list, require a stack of ammunition once during production, to fill the units supplies and make it combat ready. Rather than having multiple copies of the same unit but with different ammunition requirements for various patterns, consult the following table for ammunition requirement depending on the units main weaponry.
\begin{longtable}{l l l}\toprule
     Weapon Type & Example & Requirement  \\\midrule\endhead
     Small Arms & H.Stubber, Multilaser, Bolter & GG\\ \addlinespace
     Cannons & Battlecannon, Earthshaker & Shells \\\addlinespace
     High-Power Las & Las-Destructor, Lance-Battery & Rare Minerals \\\addlinespace
     Flamer & Heavy Flamer, Inferno-Cannon & Promethium(2) \\\addlinespace
     Missile Launcher & Hunter-Killer, Manticore & Missiles(2)\\\addlinespace
     Plasma/Melta & Plasmacannon, Plasma-Broadside & Rare Minerals, Promethium \\\bottomrule
\end{longtable}
